\section{Model Reduction of Fluid Flow} \label{sec:mor_skew}
A straight-forward model reduction of \eqref{eq:3.8} and \eqref{eq:3.18} does not, in general, preserve symmetries and conservation laws, presented in \Cref{sec:skew}. In this section we discuss how to exploit the discrete skew-symmetric structure of \eqref{eq:3.13} and \eqref{eq:3.21} to recover conservation of mass, momentum, and energy at the level of the reduced system.

Let $V_{\mathbf r}$, $V_{\mathbf r \mathbf u}$, and $V_{\mathbf u_i}$ be the reduced bases for the snapshots of $\mathbf r$, $R \mathbf u$, and $\mathbf u_i$, respectively. For the one dimensional case, the subscript ``$i$'' is omitted and for an incompressible fluid, $V_{ \mathbf r}$ and $V_{\mathbf r \mathbf u}$ are not computed. For the purpose of simplicity, we assume that all bases have the size $k$. We seek to project $S_{\mathbf u}$ and $S_{\mathbf r \mathbf u}$ onto the reduced space, such that the projection preserves the skew-symmetric property. The projected operators, using a Galerkin projection, read
\begin{equation} \label{eq:4.1}
	S^r _{\mathbf u} = V_{ \mathbf u_i}^T S _{\mathbf u} V_{ \mathbf u_i}, \quad i=1,\dots,d,
\end{equation}
and
\begin{equation} \label{eq:4.2}
	S^r_{\mathbf r ,\partial_t} =V_{\mathbf r \mathbf u}^T  S_{\mathbf r ,\partial_t} V_{\mathbf u}, \quad S^r _{\mathbf r \mathbf u} = V_{\mathbf r \mathbf u}^T  S _{\mathbf r \mathbf u} V_{\mathbf u}.
\end{equation}
Note that $S^r_{\mathbf r ,\partial_t}$ is not computed explicitly. It is clear that $S^r _{\mathbf u}$ is already in a skew-symmetric form. On the other hand, $S^r_{\mathbf r ,\partial_t}$ and $S^r _{\mathbf r \mathbf u}$ are not, in general, skew-adjoint and skew-symmetric, respectively. This can be ensured, however, by requiring $V_{\mathbf r \mathbf u} = V_{\mathbf u}$. We denote such a basis by $V_{\mathbf r \mathbf u, \mathbf u}$.Using \eqref{eq:4.1} and \eqref{eq:4.2}, a Galerkin projection of the momentum equation in \eqref{eq:3.13} and the governing equations for a compressible fluid in \eqref{eq:3.21} take the form
\begin{equation} \label{eq:4.3}
	\frac{d}{dt} {\mathbf u^r}_i + S^r_{\mathbf u} \mathbf u^r_i + V_{\mathbf u_i} ^T D_i \mathbf p = \sum_{j=1}^d V_{k_3, \mathbf u_i}^T D_j T_{ij}(V_{ \mathbf u_i} \mathbf u^r_i), \quad i=1,\dots,d,
\end{equation}
and
\begin{equation} \label{eq:4.4}
\left\{
\begin{aligned}
	& \frac{d}{dt} \mathbf r^r + \sum_{i=1}^k V^T_{\mathbf r}DU_iV_{\mathbf r}\mathbf r^r = 0, \\
	& S^r_{\mathbf r ,\partial_t} + S^r _{\mathbf r \mathbf u} \mathbf u^r + V_{\mathbf r \mathbf u, \mathbf u}^T D V_{\mathbf p} \mathbf p^r = V_{\mathbf r \mathbf u, \mathbf u}^T D T, \\
	&\frac{1}{\gamma -1} \frac{d}{dt} \mathbf p^r + \frac{\gamma}{\gamma -1} V_{\mathbf p}^T D U V_{\mathbf p} \mathbf p^r - V_{\mathbf p}^T UD V_{\mathbf p} \mathbf p^r = - V_{\mathbf p}^T UDT + V_{\mathbf p}^T D(UT - \mathbf \phi),
\end{aligned}
\right.
\end{equation}
respectively. Note that in \eqref{eq:4.4}, dependency of $T$ on $V_{\mathbf r \mathbf u , \mathbf u}$ is not shown for abbreviation. In \eqref{eq:4.3} and \eqref{eq:4.4}, $D_i$ is always multiplied from the left with a basis matrix or a diagonal matrix. Therefore, the telescoping sum, discussed in Condition 2 in \Cref{sec:skew.1}, cannot be used to show conservation of mass and momentum. However, POD preserves linear properties of snapshots. To demonstrate this, let the overscript ``\textasciitilde'' denote the representation of a reduced variable in the high-fidelity space. An approximated variable, e.g. density, can be represented as a linear combination of some snapshots as $\mathbf r \approx \tilde{\mathbf r} = \sum_{i=1}^k c_i \mathbf r_i$, for some snapshots $\mathbf r_i$ and some coefficients $c_i \in \mathbb R$, for $i=1,\dots,k$. Conservation of mass, evaluated by $\tilde{\mathbf r}$, reads
\begin{equation} \label{eq:4.5}
	\frac{d}{dt} \mathbf 1^T \tilde {\mathbf r} = \sum_{i=1}^k c_i  \left( \mathbf 1^T \frac{d}{dt} \mathbf r_i \right) = - \sum_{i=1}^k c_i  \left( \mathbf 1^T DR_i\mathbf u_i \right) = 0,
\end{equation}
where we used the fact that $\mathbf 1^T D = \mathbf 0^T$. Similarly, we recover conservation of momentum
\begin{equation} \label{eq:4.6}
\begin{aligned}
	\frac{d}{dt}(\tilde {\mathbf r}^T \tilde{\mathbf u}) &= \frac{1}{2} \frac{d}{dt}(\tilde{\mathbf r}^T \tilde{\mathbf u}) + \frac{1}{2} \left( \tilde{ \mathbf r }^T \frac d{dt} \tilde{ \mathbf u } + \tilde {\mathbf u}^T \frac{d}{dt} \tilde {\mathbf r} \right)\\
	&= \sum_{i,j=1}^k d_i c_j \left( \mathbf u_i^T \frac{d}{dt} \mathbf r_j + \left( \mathbf r_j^T \frac d{dt} \mathbf u_i +\mathbf u_i^T \frac{d}{dt} \mathbf r_j \right)\right) = 0.\\
\end{aligned}
\end{equation}
Here, $\tilde{\mathbf u} = \sum_{i=1}^k d_i \mathbf u_i$, for some snapshot $\mathbf u_i$ and coefficients $d_i \in \mathbb R$. Denoting by $\{ R \mathbf u\}^r$ the reduced representation of $R\mathbf u$ in basis $V_{\mathbf r \mathbf u , \mathbf u}$, the evolution of kinetic energy is expressed as
\begin{equation} \label{eq:4.7}
	\begin{aligned}
	\frac{d}{dt}\left( \frac 1 2 \tilde{\mathbf u}^T \tilde{ R } \tilde {\mathbf u} \right) &= \frac{d}{dt}\left( \frac 1 2 {\mathbf u^r}^T V_{\mathbf r \mathbf u , \mathbf u}^T V_{\mathbf r \mathbf u , \mathbf u} \{ R \mathbf u\}^r \right) = \frac{d}{dt}\left( \frac 1 2 {\mathbf u^r}^T \{ R \mathbf u\}^r \right) \\
	&= \frac 1 2 \left( {\mathbf u^r}^T \frac{d}{dt} \{ R \mathbf u \}^r + \{ R \mathbf u \}^r \frac{d}{dt} \mathbf u^r \right) \\
	&= \frac 1 2 \left( {\mathbf u^r}^T V_{\mathbf r \mathbf u , \mathbf u}^T V_{\mathbf r \mathbf u , \mathbf u} \frac{d}{dt} \{ R \mathbf u \}^r + \{ R \mathbf u \}^r \frac{d}{dt} V_{\mathbf r \mathbf u , \mathbf u}^T V_{\mathbf r \mathbf u , \mathbf u} \mathbf u^r \right) \\
	&= {\mathbf u^r}^T S^r_{\mathbf r, \partial_t} \mathbf u^r = { \mathbf u^r }^T V_{\mathbf r \mathbf u , \mathbf u} D V_{\mathbf p} \mathbf P^r + { \mathbf u^r }^T V_{\mathbf r \mathbf u , \mathbf u}^T D T.
	\end{aligned}
\end{equation}
In the missing steps in the last line, skew-symmetry of $S_{\mathbf r \mathbf u}^r$ is used. Note, that only the reduced pressure and the viscous term contribute to the evolution of kinetic energy. Furthermore, the quantity $ \frac 1 2 {\mathbf u^r}^T \{ R \mathbf u\}^r$ is the kinetic energy associated with the reduced system \eqref{eq:4.4}, approximating the kinetic energy of the high-fidelity system \eqref{eq:3.21}, and is a quadratic form with respect to the reduced variables. Conservation of kinetic energy for \eqref{eq:4.3} follows similarly. It is  straight-forward to check that
\begin{equation} \label{eq:4.8}
	\frac{d}{dt} \left( \frac{1}{\gamma - 1} \mathbf 1^T \tilde{\mathbf p} + \frac{1}{2} \tilde{\mathbf u}^T \tilde R \tilde{\mathbf u} \right) = 0,
\end{equation}
i.e., the total energy is conserved. We immediately recognize that $ \mathbf p^r /(\gamma - 1)$ is the internal energy of the reduced system. However, the total internal energy of \eqref{eq:4.4} is a weighted sum, $ b^T\mathbf p^r /(\gamma - 1)$, with $b = V_{\mathbf p}^T \mathbf 1$ which is an approximation of the total internal energy in \eqref{eq:3.21}. From \eqref{eq:4.5}, \eqref{eq:4.6}, \eqref{eq:4.7}, and \eqref{eq:4.8} we conclude the following proposition.
\begin{proposition}
	The loss in the mass, momentum and energy associated with the model reduction in \eqref{eq:4.4} is constant in time, and therefore, bounded.
\end{proposition}

\subsection{Assembling Nonlinear Terms and Time Integration}
Nonlinear terms that appear in \eqref{eq:4.3} and \eqref{eq:4.4} are of quadratic nature. These terms can be evaluated exactly using a set of precomputed matrices as proposed in \cite{Benner2018}. As an example, consider
\begin{equation}
	S_{\mathbf u}^r = V^T_{\mathbf u} ( DU + UD ) V^T_{\mathbf u}.
\end{equation}
We write $U$ as a linear combination of matrices as $U = \sum_{j=1}^k \mathbf u ^r_j U_j$, where $\mathbf u ^r_j$ is the $j$th component of $\mathbf u ^r$, and $U_j$ contains the $j$th column of $V_{\mathbf u}$ on its diagonal. It follows
\begin{equation}
	S_{\mathbf u}^r = \sum_{j=1}^k \mathbf u ^r_j \left( V^T_{\mathbf u} ( DU_j + U_jD ) V^T_{\mathbf u} \right).
\end{equation}
The matrices $V^T_{\mathbf u} ( DU_j + U_jD ) V^T_{\mathbf u}$ can be computed prior to the time integration of the reduced system. However, the form of the fully discrete system in \eqref{eq:3.27} introduces cubic and even quartic terms. In principle, the same method can be applied to assemble the nonlinear terms. However, the number of precomputed matrices grows proportional to the order of the nonlinear term. 

To accelerate assembly of the nonlinear terms we may approximately evaluate them using the discrete empirical interpolation method (DEIM). This approximation can affect the accuracy of conserved quantities in \eqref{eq:4.4}. Therefore, the accuracy of the DEIM approximation must be chosen higher than the one of POD.

To integrate \eqref{eq:4.4} in time, the fully discrete system \eqref{eq:3.27} is modified prior to model reduction, by dividing the mass and momentum equation with $\sqrt{\mathbf r}^{n+1}$. Note that since the new form is identical to \eqref{eq:3.27}, it does not affect the conserved quantities. Subsequently, a basis for $\sqrt{\mathbf r}$, denoted by $V_{\sqrt{\mathbf r}}$, is constructed. The nonlinear terms are evaluated exactly using the quadratic expansion or approximated using the DEIM.
