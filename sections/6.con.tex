\section{Conclusions} \label{sec:con}

Conservation of nonlinear invariants are not, in general, guaranteed with conventional model reduction techniques. The violation of such invariants often result in a qualitatively wrong or unstable reduced system, even when the high-fidelity system is stable. This is particularly important for fluid flow, where conservation of the energy, as a nonlinear invariant of the system, is crucial for a correct numerical evaluation.

In this paper, we discuss that conservative properties of the skew-symmetric form for fluid flow can naturally be extended to the reduced system. Conventional MOR techniques preserves the skew-symmetry of differential operator which result in the conservation of quadratic invariants at the level of the reduced system. Furthermore, the reduced system also contains quadratic invariants with respect to the reduced variables that approximates the invariants of the high-fidelity system. This results in the construction of a physically meaningful reduced system, rather than a mere couple systems of differential equations.

Numerical experiments for the incompressible and compressible Euler equation confirms conservation of mass, momentum and energy for the reduced model with the skew-symmetric discretization. In contrast, when a non-skew-symmetric form, e.g. divergence form or advective form, is considered, MOR does not necessarily yield a stable reduced system. On the other hand the skew-symmetric form consistently yields a robust reduced system over long time-integration, even when the reduced space does not represent the high-fidelity solution accurately. 

Finally, a MOR of a quasi-skew-symmetric form for the CVRC model is presented. Although this model is not in a full skew-symmetric form and an explicit Runge-Kutta method used for time-integration, we still recover a reduced model with excellent stability properties. 

\section*{Acknowledgement} The work was partially supported by AFOSR under grant FA9550-17-1-9241 and by SNSF under the grant number P1ELP2-175039.
