\section{Introduction} \label{sec:intro}

Model order reduction (MOR), and in particular reduced basis (RB) methods, has emerged as a powerful approach to cope with the complex and computationally intensive models in engineering and science. Such techniques construct a reduced ordered representation for the state of a model which accurately approximates the configuration of the system. The evaluation of this representation is then possible with considerable acceleration.

Although RB methods are successful in reducing the computational complexity of models with elliptic and parabolic partial differential equations (PDEs), MOR of systems of hyperbolic equations, or models with strong advective terms, remains a challenge. Often, such models arise from a set of invariants and conservation laws, some of which are violated by MOR which result in a qualitatively wrong, and sometimes unstable, solution.

Constructing MOR techniques and RB methods that preserve intrinsic structures has recently attracted attention \cite{doi:10.1137/17M1111991,kalashnikova2014stabilization,farhat2015structure,doi:10.1137/110836742,beattie2011structure,doi:10.1137/140978922}. Structure preservation can recover a physically meaningful reduced model, rather than a pure algebraic coupling of equations. This enforces robustness and can help with the stability of the reduced model. Preserving time-symmetries of Lagrangian, Hamiltonian, and port-Hamiltonian systems can be found in the works of \cite{Carlberg:2014ky,doi:10.1137/140978922,doi:10.1137/17M1111991,beattie2011structure,chaturantabut2016structure,gugercin2012structure}. Conserving inf-sup stability, in the context of finite element methods, can be found in \cite{farhat2015structure,ballarin2015supremizer}. Furthermore, a flux preserving model reduction for finite volume methods is presented in \cite{carlberg2018conservative}. 

Large scale simulations of fluid flows arise in a wide range of disciplines and industries. Therefore, MOR of fluid flows, specially when advective terms are dominant, is important. It is well known that conservation of the energy, specially kinetic energy, is essential for a qualitatively correct numerical integration of fluid flows. Conventional model reduction techniques often violates conservation of mass, momentum \cite{carlberg2018conservative}, or energy in fluid flows which result in an unstable reduced system, in particular for long time-integration. \edit{In \cite{barone2009stable} an entropy stable model reduction method for linear compressible flows is presented by considering an entropy-stable formulation of linearized compressible flows. Furthermore, a conservative model reduction for finite-volume models is presented in \cite{carlberg2018conservative} that conserves any quantity conserved by the finite-volume scheme. This method finds a reduced linear subspace that ensures conservation of quantities by solving an optimization problem with, generally nonlinear, equality constraints. The constrained optimization problem is solved online, and is only slightly more expensive than solving the unconstrained optimization problem associated with a typical Galerkin MOR.}

%\edit{The method discussed in \cite{carlberg2018conservative} conserves the mass and momentum for a finite volume discretization scheme. However, a method that also conserves the energy of the fluid flow is not known to the authors.}

Skew-symmetric formulation of fluid flows constructs a skew-symmetric differential operator, acting on the momentum vector field, that ensures conservation of quadratic invariants, such as energy. \edit{Combined with} centered time and space discretization schemes, typically a finite differences discretization method, they recover time-symmetries of a fluid at the discrete level. Such discretization schemes are studies comprehensively over the past few decades and can be found in the works of \cite{morinishi2010skew,morinishi1998fully,desjardins2008high,tadmor1984skew,reiss2014conservative} and the references therein.

In this paper we discuss how to preserve skew-symmetry of the differential operators at the level of the reduced system. This results in conservation of quadratic invariants. \edit{The conservation of quantities in the proposed method is guaranteed through the mathematical formulation of the reduced system, for any orthonormal reduced basis. Therefore, the offline and online computational costs of this method is comparable with conventional MOR techniques. However, other conservative model reduction methods, e.g. \cite{carlberg2018conservative}, often require solving multiple nonlinear optimization problems to ensure conservation which can increase the computational costs.} Furthermore, we show that the reduced system, as a system of coupled differential equations, contains quadratic invariants and an associated energy which approximates the energy of the high-fidelity system. Therefore, a proper time stepping scheme preserves the reduced representation of the energy, and therefore, the loss in energy due to model reduction remains constant in time. Furthermore, we demonstrate, through numerical experiments, that a quasi-skew-symmetric form of fluid flow, i.e. a formulation where only spacial differential operators are in a skew-symmetric form, offer remarkable stability properties in terms of MOR. This allows an explicit time-integration to be utilized while recovering robustness of skew-symmetric forms at the reduced level.

The rest of this paper is organized as follows. In \Cref{sec:mor} we summarize the theory on MOR and introduce the proper orthogonal decomposition (POD) as a conventional RB method. We discuss skew-symmetric and conservatives methods for compressible and incompressible fluid flows in \Cref{sec:skew}. Conservative and energy-preserving model reduction of fluid flows is discussed in \Cref{sec:mor_skew}. We evaluate the performance of the method through numerical simulations of incompressible and compressible fluid flow in \Cref{sec:res}. We also apply the method to construct a reduced system for the continuous variable resonance combustor, a one dimensional reaction-diffusion model for a rocket engine. Finally, we present conclusive remarks in \Cref{sec:con}.
